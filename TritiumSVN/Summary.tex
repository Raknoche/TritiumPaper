\section{Summary}

We have characterized the electron recoil response of the LUX dark matter experiment with a tritium calibration source. The large dataset, high event purity, and the simple nature of the decay provide a powerful tool to study the detector and to investigate the fundamental properties of LXe as a particle detection medium for WIMP searches. 

We find strong evidence in support of the combined energy model for ER events in the WIMP energy range, and we report new measurements of the light and charge yields, the average recombination, and the fluctuations in the recombination as a function of energy. We have determined that the width of the ER band in LUX is driven by fluctuations in the number of detected S1 photons. We find a small number of outlier events far below the ER band centroid out of 170,000 fiducial tritium decays, consistent with background expectations in this dataset.

The results presented here are used in an improved analysis of the Run 3 WIMP search data to determine the location and width of the LUX ER band and to measure the fiducial volume~\cite{lux-reanalysis}. Additional tritium data has also been collected in support of the on-going LUX Run4 WIMP search and is presently under analysis.