\section{Introduction}

LUX is a large dual-phase liquid xenon (LXe) time projection chamber (TPC) located at the 4850' level of the Sanford Underground Research Facility (SURF) in Lead, South Dakota. LUX searches for WIMP dark matter interactions by detecting both scintillation (S1) and charge signals (S2) from particle interactions in the LXe. Electron-recoil (ER) and nuclear-recoil (NR) events are distinguished by the ratio of the charge and light signals (S2/S1). Results from the first LUX science run (Run 3) were first reported in Ref. \cite{LUX_PRL}. An improved analysis of the Run 3 data is reported in Ref. \cite{lux-reanalysis}.

In the WIMP region of interest ($\sim$1 - 8 keVee), the LUX rejects external gamma backgrounds primarily through passive self-shielding, achieving an event rate of \fixit{3.6 mDRU between 0.9 and 5.3 keVee in the 118 kg fiducial volume}. (One mDRU is $10^{-3}$ events/keVee/kg/day). The power of self-shielding derives from a combination of two effects: the mismatch between the energy range of interest ($\sim$few keV) and the energy scale of penetrating gammas ($>$  several hundred keV), and the small probability for a gamma to cross the active volume while scattering only once. In LUX this effect is substantial due to the large active mass (270 kg) and favorable aspect ratio (50 cm diameter and 60 cm height). The LUX TPC reconstructs the three coordinates of the event vertex with a spatial precision of about 1~cm. 

Self-shielding also presents a challenge for detector calibration, since external gamma sources such as \cssrc or \thsrc are, by design, unable to produce a useful rate of single-scatter calibration events in the fiducial volume at low energy. As an alternative, two internal calibration sources that defeat self-shielding have been deployed in LUX, the first based upon \krsrc\cite{Kastens:2009rt, baudis}, and the second based upon tritium ($^{3}$H). These internal sources are dissolved into the LXe target material, allowing the response of the detector to be studied with a large sample of spatially uniform events.

\krsrc is a source of 9.4 keV and 32.1 keV internal conversion electrons separated in time by an intermediate state with a half life of 154 ns\cite{83Kr_HalfLife_1}\cite{83Kr_HalfLife_2}.  Due to its distinct line-source energy spectrum, \krsrc is well-suited for monitoring and correcting the spatial and time dependence of the S1 and S2 signals. However, because both \krsrc electrons are above the energy range of interest for dark matter, and because the S2 signals from the two electrons generally overlap in time,  \krsrc is less useful for characterizing the S2/S1 electron-recoil discriminant variable or for studying the detector threshold.

In this article we report results from the calibration of LUX with an internal tritium source, which plays a complementary role to the \krsrc source. Tritium is a single-beta emitter with a Q value of 18.6 keVee \cite{Tritium_Q}. Its mean energy is 5.6 keVee \cite{Tritium_Mean} with a broad peak at 3.0 keVee.,and 75\% of its beta decays are below 8 keVee \cite{Tritium_Eq}. These characteristics make it a nearly ideal source for studying the ER response of LUX in the dark matter energy range. Unlike \krsrc, however, tritium is long-lived, with a half-life of 12.3 years\cite{Tritium_halflife_all}, so the tritium must be removed by purification. In addition, tritium must be introduced into the detector in a manner which will not impair the charge or light collection properties of the detector. 

We use tritiated methane (CH$_3$T) as the host molecule to deliver tritium activity into LUX. Compared to molecular tritium (T$_2$), methane has several desirable properties. First, it is chemically inert, so it does not adsorb onto surfaces as the T$_2$ molecule is known to do, and it is consistent with maintaining good charge transport in liquid xenon. Secondly, because the LUX TPC contains teflon (PTFE) and polyethylene (PE) components, diffusion of tritium activity into the detector parts is a concern insofar that such activity may later contaminate the LXe during WIMP search data taking. Methane mitigates this risk due to its larger molecular size, the diffusion constant times solubility of methane is ten times smaller than T$_2$ in these plastics, at least at room temperature where good data is available. This reduces the risk of tritium backgrounds during the WIMP search. Ultimately we carried out a series of bench-top tests of the CH$_3$T injection and removal procedure using samples of LUX plastics immersed in LXe. These tests indicated that the risk of re-contamination was small as long as the amount of injected activity is modest.

 its diffusion constant (D) times solubility (K) is ten times smaller in materials such as teflon (PTFE) and polyethylene (PE)\cite{miyake:1983}, at least at room temperature where good data is available, mitigating the problem of back-diffusion of activity into the liquid xenon after purification;  In developing and implementing the tritiated methane source for LUX, we took care to understand and control the removal of the activity from the detector, beginning with a series of demonstration experiments with a small detector. We also quantified the absorption and back-diffusion of tritiated methane from the plastics used in the LUX detector. The results of these investigations, and our analysis of the risks associated with the tritium injection, are described in the appendix of this article.

An initial tritium dataset of $\sim$ 7,000 fiducial events was obtained by LUX in August of 2013, and the results were reported in Ref. \cite{lux-results}. Subsequently, in December 2013, we injected additional activity with a higher rate and obtained a fiducial tritium dataset of over 150,000 events. This dataset is used to calibrate the LUX ER band in Ref. \cite{lux-reanalysis}. Except where otherwise noted, in this article we report results from the larger December 2013 dataset.