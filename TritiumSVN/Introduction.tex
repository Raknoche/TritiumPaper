\section{Introduction}

The Large Underground Xenon experiment (LUX) is a WIMP search located at the 4850' level of the Sanford Underground Research Facility (SURF) in Lead, South Dakota~\cite{lux-nim}. LUX detects particle interactions in liquid xenon (LXe) via scintillation (S1) and ionization charge (S2) signals. The LXe is instrumented as a dual-phase time projection chamber (TPC), providing an energy measurement, position information in three dimensions, and single-scatter event identification. Electron-recoil (ER) and nuclear-recoil (NR) interactions are distinguished by the ratio of the charge and light signals (S2/S1). Results from the first LUX science run (Run 3) were first reported in Ref.~\cite{lux-prl}. An improved analysis of the Run 3 data is reported in Ref.~\cite{lux-reanalysis}.

To calibrate the ER response of LUX, external gamma sources such as \cssrc are occasionally employed, but such sources are unable to produce a useful rate of fiducial single-scatter events in the WIMP energy range of interest due to self-shielding. Therefore the ER response is monitored and calibrated primarily with electron-emitting radioisotopes that can be dissolved in the LXe. Two such sources,  \krsrc~\cite{Kastens:2009pa, Baudis} and tritium ($^{3}$H), have been deployed, both providing a large sample of spatially-uniform events. In this article we report results from the calibration of LUX with tritium, a single-beta emitter with a Q-value of 18.6~keV electron-equivalent\footnote{ER events and NR events generally have different energy scales in LXe. In this article we interpret all events using the ER energy scale.}~\cite{Tritium_Q}. Neutron sources and a neutron generator are also employed by LUX to study the response to NR events~\cite{DD-paper}.

The tritium beta spectrum is well known both theoretically and experimentally. It has a broad peak at 2.5~keV and a mean energy of 5.6~keV~\cite{Tritium_Mean,Tritium_Eq,Drexlin:2013lha}. 64.2\% of the decays occur between 1 and 8~keV, the energy range of interest for WIMP searches in LUX. These characteristics make it an ideal source for studying the ER response of the detector.  $\rm ^{83m}$Kr, which emits 9.4~keV and 32.1~keV internal conversion electrons, is well suited for routine monitoring and for correcting the spatial and temporal variations of the S1 and S2 signals, but is less useful for studies of the S2/S1 ER discrimination variable because both conversion electrons are above the dark matter energy range, and because the S2 signals from the two electrons generally overlap in the detector due to the short half-life of the intermediate state (154~ns). We note that the most important background in LUX is due to Compton scatters, and such events are expected to have similar properties to beta decays in the tritium energy range~\cite{NEST_2013}. 

We use tritiated methane (CH$_3$T) as the host molecule to deliver tritium activity into LUX. Compared to molecular tritium (T$_2$), CH$_3$T has several advantages. It does not adsorb onto surfaces like the T$_2$ molecule, and it does not interfere with charge transport in LXe. Also, because of its 12.3 year half-life, tritium must be removed from the detector by purification, and methane is amenable to chemical removal with standard noble gas purifiers~\cite{Dobi_CH4}. Note, however, that diffusion of tritium activity into plastic detector components during the calibration is an important concern, since that activity may later re-contaminate the LXe during the WIMP search runs.  In this respect, CH$_3$T is preferable over T$_2$ due to its larger molecular size and lower diffusion constant and solubility~\cite{miyake:1983}. We investigated the CH$_3$T contamination risk empirically with a series of bench-top tests prior to the first injection into LUX. These tests, which are described in Appendix \ref{sec:appendix1}, demonstrated that the injection and removal could be done without undue risk to the experiment. 

An initial tritium dataset of $\sim$7,000 fiducial events was obtained in August of 2013, and the results were reported in Ref.~\cite{lux-prl}. Subsequently, in December 2013, we injected additional activity with a higher rate and obtained a fiducial tritium dataset of 170,000 events. This dataset is used to characterize the LUX ER band in Ref.~\cite{lux-reanalysis}. Except where otherwise noted, in this article we report results from the larger December 2013 dataset.