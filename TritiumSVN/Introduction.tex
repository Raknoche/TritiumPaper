\section{Introduction}

LUX is a WIMP search experiment located at the 4850' level of the Sanford Underground Research Facility (SURF) in Lead, South Dakota. LUX detects particle interactions in liquid xenon (LXe) via scintillation (S1) and charge (S2) signals. The LXe in LUX is instrumented as a dual-phase time projection chamber (TPC), providing an energy measurement, event topology, and position information in three dimensions. Electron-recoil (ER) and nuclear-recoil (NR) events are distinguished by the ratio of the charge and light signals (S2/S1). Results from the first LUX science run (Run 3) were first reported in Ref. \cite{lux-prl}. An improved analysis of the Run 3 data is reported in Ref. \cite{lux-reanalysis}.

LUX is monitored and calibrated primarily with beta sources that can be dissolved in the LXe. Two such sources,  \krsrc\cite{Kastens:2009rt, Baudis} and tritium ($^{3}$H), have been deployed in LUX, both providing a large sample of spatially uniform events. In this article we report results from the calibration of LUX with tritium, a single-beta emitter with a Q value of 18.6 keV\cite{Tritium_Q}. The mean decay energy of tritium is 5.6 keV, with a broad peak at 2.5 keV, and 75\% of its decays are below 8 keV\cite{Tritium_Mean,Tritium_Eq}. These characteristics make tritium an ideal source for studying the ER response of LUX in the dark matter energy range ($\sim$1-8 keV (electron equivalent)). $^{83m}$Kr, which is a source of 9.4 keV and 32.1 keV internal conversion  (IC) electrons, is well suited for monitoring and correcting the spatial and time variations of the S1 and S2 signals, but is less useful for studies of the S2/S1 ER discrimination variable because both IC electrons are above the dark matter energy range, and because the S2 signals from the two electrons generally overlap in the detector. External gamma sources, such as \cssrc or $^{232}$Th, are unable to produce a useful rate of fiducial single-scatter calibration events in LUX in the WIMP region of interest ($\sim$1-8 keV) due to self-shielding. 

We use tritiated methane (CH$_3$T) as the host molecule to deliver tritium activity into LUX. Compared to molecular tritium (T$_2$), CH$_3$T has several advantages. It does not adsorb onto surfaces like the T$_2$ molecule, and it does not interfere with charge transport in LXe. Also, because of its 12.3 year half-life, tritium must be removed from the detector by purification, and methane is amenable to chemical removal with standard noble gas purifiers\cite{Dobi_CH4}. Note, however, that diffusion of tritium activity into plastic detector components during the calibration is an important concern, since that activity may later re-contaminate the LXe during the WIMP search runs.  In this respect, CH$_3$T is preferable over T$_2$ due to its larger molecular size and lower diffusion constant and solubility\cite{miyake:1983}. We investigated the CH$_3$T contamination risk empirically with a series of bench-top tests prior to the first injection into LUX. These tests, which are described in Appendix \ref{sec:appendix1}, demonstrated that the injection and removal could be done without undue risk to the experiment. 

An initial tritium dataset of $\sim$7,000 fiducial events was obtained by LUX in August of 2013, and the results were reported in Ref. \cite{lux-prl}. Subsequently, in December 2013, we injected additional activity with a higher rate and obtained a fiducial tritium dataset of over \fixit{180,000 events - in the caption of Fig 11 we say 130,000 events. Why?}. This dataset is used to characterize the LUX ER band in Ref. \cite{lux-reanalysis}. Except where otherwise noted, in this article we report results from the larger December 2013 dataset.