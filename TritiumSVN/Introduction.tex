\section{Introduction}

LUX is a dual-phase liquid xenon (LXe) time projection chamber (TPC) located at the 4850' level of the Sanford Underground Research Facility (SURF) in Lead, South Dakota. LUX searches for WIMP dark matter interactions by detecting scintillation (S1) and charge (S2) signals from particle interactions in the LXe. Electron-recoil (ER) and nuclear-recoil (NR) events are distinguished by the ratio of the charge and light signals (S2/S1). Results from the first LUX science run (Run 3) were first reported in Ref. \cite{LUX_PRL}. An improved analysis of the Run 3 data is reported in Ref. \cite{lux-reanalysis}.

%In the WIMP region of interest ($\sim$1 - 8 keVee), LUX rejects external gamma backgrounds primarily through self-shielding, achieving an event rate of \fixit{3.6 mDRU\footnote{One mDRU is $10^{-3}$ events/keVee/kg/day} between 0.9 and 5.3 keVee in the 118 kg fiducial volume}.  The LUX TPC reconstructs the three coordinates of the event vertex with a spatial precision of about 1~cm,  allowing events near the edge of the detector to be rejected. 

LUX is monitored and calibrated with beta sources that can be dissolved in the LXe. External gamma sources, such as \cssrc or \thsrc, are unable to produce a useful rate of fiducial single-scatter calibration events in the WIMP region of interest ($\sim$1 - 8 keVee) due to self-shielding. This effect is substantial in LUX because of the large active mass (270 kg) and favorable aspect ratio (50 cm diameter and 60 cm height). Two internal calibration sources,  \krsrc\cite{Kastens:2009rt, baudis} and tritium ($^{3}$H), have been deployed in LUX, both providing a large sample of spatially uniform events. \krsrc is a source of 9.4 keV and 32.1 keV internal conversion  (IC) electrons and is well-suited for monitoring and correcting the spatial and time variations of the S1 and S2 signals. \krsrc is unable, however, to characterize the S2/S1 electron-recoil discriminant variable, or to probe the detector threshold, because both IC electrons are above the dark matter energy range, and because the S2 signals from the two electrons generally overlap in the detector.

In this article we report results from the calibration of LUX with tritium, a single-beta emitter with a Q value of 18.6 keVee \cite{Tritium_Q}. \fixit{ Its mean decay energy is 5.6 keVee \cite{Tritium_Mean} with a broad peak at 3.0 keVee. 75\% of its beta decays are below 8 keVee \cite{Tritium_Eq}.} These characteristics make it an ideal source for studying the ER response of LUX in the dark matter energy range. 

Tritium must be removed from the detector by purification due to its long half-life (12.3 years\cite{Tritium_halflife_all}), and should only be introduced into the experiment in a manner that will not impair the detector's charge or light collection properties. We use tritiated methane (CH$_3$T) as the host molecule to deliver tritium activity into LUX. Compared to molecular tritium (T$_2$), CH$_3$T has several advantages. It does not adsorb onto surfaces like the T$_2$ molecule, and it does not interfere with charge transport in LXe. Also, diffusion of tritium activity into the plastic detector parts is an important concern, since such activity may later re-contaminate the LXe during the WIMP search runs.  This risk is mitigated by CH$_3$T due to its larger molecular size and lower diffusion constant and solubility. To study the contamination risk of the CH$_3$T calibration, we performed a series of experiments with a small LXe detector and samples of LUX plastics. These tests demonstrated that the injection and removal could be done without undue risk to the experiment. These experiments are described in Appendix \ref{appendix}.

An initial tritium dataset of $\sim$ 7,000 fiducial events was obtained by LUX in August of 2013, and the results were reported in Ref. \cite{lux-results}. Subsequently, in December 2013, we injected additional activity with a higher rate and obtained a fiducial tritium dataset of over 150,000 events. This dataset is used to calibrate the LUX ER band in Ref. \cite{lux-reanalysis}. Except where otherwise noted, in this article we report results from the larger December 2013 dataset.