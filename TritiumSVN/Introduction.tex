\section{Introduction}

LUX is a large dual-phase liquid xenon (LXe) time projection chamber (TPC) designed to search for WIMP dark matter with an active target mass of 270 kg. LUX detects both scintillation and charge signals from particle interactions in the LXe. Electron recoil (ER) and nuclear recoil (NR) events are distinguished by the ratio of charge and light signals (S2/S1), and the TPC reconstructs all three coordinates of the event vertex. Results from the first LUX science run (Run 3) were initially reported in Ref. \cite{LUX_PRL}. A recent re-analysis of the Run 3 data which takes advantage of additional calibrations and an improved understanding of the detector is reported in Ref. \cite{lux-reanalysis}.

In the WIMP region of interest ($\sim$1 - 8 keVee), LUX rejects external gamma backgrounds primarily through the passive action of self-shielding, achieving an event rate of XXX mDRU in the fiducial volume.  Self-shielding is powerful because gammas in the energy range of interest cannot penetrate to the fiducial volume, while gammas with penetrating power ($> \sim$ 300 keV) are unlikely to scatter only once and deposit only a few keV of energy.

While self-shielding makes the detector insensitive to gamma backgrounds, it also presents a challenge for conventional detector calibration techniques for ER events, because  external gamma sources such as \cssrc or \thsrc are unable to produce a useful rate of single-scatter calibration events in the fiducial volume at low energy. As an alternative, two internal calibration sources that defeat the detectors' self-shielding have been developed and deployed in LUX; the first based upon \krsrc\cite{Kastens:2009rt}, and the second based upon tritium ($^{3}$H). Unlike external sources, these internal sources are dissolved into the LXe target material, allowing the response of the detector to be studied with a large and spatially uniform sample of events.

\krsrc is a source of two internal conversion electrons with energies of 9.4 keVee and 32.1 keVee separated in time by an intermediate state with a half life of 154 ns\cite{83Kr_HalfLife_1}\cite{83Kr_HalfLife_2}. Because it produces two lines in the energy spectrum, \krsrc is suitable for tracking the spatial and time dependence of the S1 and S2 signals. However, because both \krsrc electrons are above the energy range of interest for dark matter, and because the S2 signals from the two electrons generally overlap,  \krsrc is less useful for characterizing the electron recoil (ER) band of the S2/S1 discriminant or for studying the detector threshold.

In this article we report results from the LUX tritium source, which plays a complementary role to the \krsrc source. Tritium is a single-beta emitter, with a Q value of 18.6 keVee \cite{Tritium_Q}. Its spectral mean beta energy is 5.6 keVee \cite{Tritium_Mean} with a peak at 3.0 keVee. 75\% of its beta decays are below 8 keVee \cite{Tritium_Eq}. These characteristics make it a nearly ideal source for studying the ER response of the detector in the dark matter energy range. Unlike \krsrc, tritium is long-lived, with a half-life of 12.3 years\cite{Tritium_halflife_all}, so the tritium must be removed from the LXe by purification. In addition, tritium must be introduced into the detector in a manner which will not impair the charge or light collection properties of the detector. This is less of a concern with \krsrc since krypton is noble.

We use tritiated methane (CH$_3$T) as the host molecule to deliver the activity into LUX. Methane has several desirable chemical and physical properties compared to T$_2$: first, its diffusion constant (D) times solubility (K) at room temperature is ten times smaller in common LUX materials such as teflon (PTFE) and polyethylene (PE)\cite{miyake:1983}, mitigating the problem of back-diffusion of activity into the liquid xenon after purification; it is chemically inert, so it is not expected to adhere to surfaces (as the T$_2$ molecule is known to do), and it is consistent with maintaining good charge transport in liquid xenon. In developing and implementing the tritiated methane source for LUX, we took considerable care to understand and control the removal of the activity from the detector, beginning with a series of demonstration experiments with a small detector. We also attempted to quantify the absorption and back-diffusion of tritiated methane from the plastics used in the LUX detector. The results of these investigations, and our analysis of the risks associated with the tritium injection, are described in the appendix.

An initial tritium dataset of $\sim$ 7,000 fiducial events was obtained by LUX in August of 2013, and the results were reported in Ref. \cite{lux-results}. Subsequently, in December 2013, we injected additional activity with a higher initial rate and obtained a fiducial tritium dataset of over 150,000 events. This dataset is used to calibrate the LUX ER band in Ref. \cite{lux-reanalysis}. Except where otherwise noted, in this article we report results from the larger December 2013 dataset.