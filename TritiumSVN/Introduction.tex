\section{Introduction}

The LUX collaboration recently reported results from its first underground science run, placing new constraints on WIMP dark matter with masses between 6~GeV and 1~TeV\cite{lux_prl}. LUX is a large dual-phase liquid xenon (LXe) time projection chamber (TPC) with an active mass of ??270?? kg. The primary scintillation light from particle interactions (S1) is collected by two arrays of photomultiplier tubes (PMTs) at the top and bottom of the detector, and the charge signal is converted to light via secondary scintillation at the anode (S2). Measurement of both S1 and S2 allows the event to be located in all three dimensions, and allows discrimination between nuclear recoil (NR) events from electron recoil (ER) events via the ratio (S2/S1).

One of the primary advantages of the liquid TPC technology is its high efficiency for the rejection of external gamma backgrounds via self-shielding. This has enabled the exploration of three orders of magnitude in WMP-nucleon cross-section over the last decade. On the other hand, self-shielding also reduces the effectiveness of external gamma sources such as \cssrc or \thsrc for detector calibration purposes, particularly in the center of the detector and at the low energies relevant for dark matter. In the case of LUX, external gamma source are unable to produce a useful rate of ER calibration events in the fiducial region.  

To address this issue, internal calibration sources that can be dissolved into the liquid xenon and thereby defeat its self-shielding have been developed\cite{mckinsey,baudis}. LUX has deployed two such internal calibration sources; the first based upon \krsrc, and the second based upon tritium ($^{3}$H). \krsrc is a source of two internal conversion electrons at energies of 9.4 keVee and 32.1 keVee separated in time by an intermediate state with a half life of 154 ns. Because it produces two lines in the energy spectrum, \krsrc is well adapted for tracking the spatial and time dependence of the S1 and S2 signals. However, because both \krsrc electrons are above the energy range of interest for dark matter (1.3 - 8 levee), and because the S2 signals from the two electrons generally overlap with each other in the detector,  \krsrc is less useful for constraining the electron recoil (ER) band of the S2/S1 discriminant. 

In this article we describe the development and use of the LUX tritium source, which plays a complementary role to the \krsrc source. Unlike \krsrc, tritium is a single-beta emitter, with a Q value of 18.6 keVee. Its spectral maximum is at 2.5 keVee, and 75\% of its beta decays are below 8 keVee. This allows the detector's ER band to be precisely characterized in a short calibration run without saturating the DAQ. In addition, the tritium spectrum has a finite decay rate extending down to zero keVee, allowing the threshold response of the detector to be studied.

Unlike \krsrc, however, tritium is long-lived, with a half-life of 12.3 years, compared to 1.8 hours for \krsrc. So while the \krsrc is naturally removed from the detector after roughly a day, the tritium must be removed from the liquid xenon by purification. Secondly, tritium must be introduced into the detector in a manner which will not impair the charge or light collection properties of the detector. This is less of a concern with \krsrc, because it can be passed through the LUX getter prior to entering the detector owing to its noble nature. 

Tritiated methane (CH$_3$T) was chosen as the appropriate host molecule to deliver the activity into LUX. Methane has several desirable chemical and physical properties compared to T$_2$: first, its diffusion constant (D) times solubility (K) is ten times smaller in common LUX materials such as teflon (PTFE) and polyethylene (PE), mitigating the problem of back-diffusion of activity into the liquid xenon after purification; it is chemically inert, so it is not expected to adhere to surfaces (as the T$_2$ molecule is known to do), and it is consistent with good charge transport in liquid xenon.

Our goal was to develop a (CH$_3$T) source which could be safely injected into LUX and removed such that any residual activity that remained (due to back-diffusion from plastics or inefficient purification) should be no more than 0.33 $\mu$Bq, which is 5\% of the LUX ER background rate design goal for a 30,000 kg$\cdot$days exposure. We desired to collect a LUX calibration dataset of $\sim$15,000 tritium events, roughly a factor of 100 larger than the number of expected ER background events in LUX. 
