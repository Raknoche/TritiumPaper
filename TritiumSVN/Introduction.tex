\section{Introduction}

LUX is a large dual-phase liquid xenon (LXe) time projection chamber (TPC) located at the 4850' level of the Sanford Underground Research Facility (SURF) in Lead, South Dakota. LUX searches for WIMP dark matter interactions by detecting scintillation (S1) and charge signals (S2) from particle interactions in the LXe. Electron-recoil (ER) and nuclear-recoil (NR) events are distinguished by the ratio of the charge and light signals (S2/S1). Results from the first LUX science run (Run 3) were first reported in Ref. \cite{Akerib:2013tjd}. An improved analysis of the Run 3 data is reported in Ref. \cite{lux-reanalysis}.

In the WIMP region of interest ($\sim$1 - 8 keVee), LUX rejects external gamma backgrounds primarily through self-shielding, achieving an event rate of \fixit{3.6 mDRU\footnote{One mDRU is $10^{-3}$ events/keVee/kg/day} between 0.9 and 5.3 keVee in the 118 kg fiducial volume}.  The LUX TPC reconstructs the three coordinates of the event vertex with a spatial precision of about 1~cm,  allowing events near the edge of the detector to be rejected. This effect is substantial in LUX because of the large active mass (270 kg) and favorable aspect ratio (50 cm diameter and 60 cm height). To calibrate the ER response of LUX we rely on internal beta sources that can be dissolved in the LXe, since external gamma sources, such as \cssrc or \thsrc, are unable to produce a useful rate of single-scatter calibration events at low energy. Two internal calibration sources,  \krsrc\cite{Kastens:2009rt, baudis} and tritium ($^{3}$H), have been deployed in LUX, allowing the ER response of the detector to be studied with a large sample of spatially uniform events.

%\krsrc is a source of 9.4 keV and 32.1 keV internal conversion electrons separated in time by an intermediate state with a half life of 154 ns\cite{83Kr_HalfLife_1}\cite{83Kr_HalfLife_2}.  Due to its distinct line-source energy spectrum, \krsrc is well-suited for monitoring and correcting the spatial and time dependence of the S1 and S2 signals. \krsrc is less useful for characterizing the S2/S1 electron-recoil discriminant variable or for studying the detector threshold because both \krsrc electrons are above the energy range of interest for dark matter, and because the S2 signals from the two electrons generally overlap in time.

In this article we report results from the calibration of LUX with the tritium source. Tritium is a single-beta emitter with a Q value of 18.6 keVee \cite{Tritium_Q}. \fixit{ Its mean energy is 5.6 keVee \cite{Tritium_Mean} with a broad peak at 3.0 keVee. 75\% of its beta decays are below 8 keVee \cite{Tritium_Eq}.} These characteristics make it an ideal source for studying the ER response of LUX in the dark matter energy range. However, tritium is long-lived, with a half-life of 12.3 years\cite{Tritium_halflife_all}, so it must be removed by purification. In addition, the tritium must be introduced into the detector in a manner which will not impair the charge or light collection properties of the detector. 

We use tritiated methane (CH$_3$T) as the host molecule to deliver tritium activity into LUX. CH$_3$T has several advantageous properties compared to molecular tritium (T$_2$). First, it does not adsorb onto surfaces like the T$_2$ molecule, and it does not interfere with charge transport in liquid xenon. Secondly, diffusion of tritium activity into the plastic detector parts is an important concern, since such activity may later contaminate the LXe during the WIMP search data runs.  This risk is mitigated by CH$_3$T compared to T$_2$ due to its larger molecular size and lower diffusion constant and solubility. We ultimately performed a series of experiments with a bench-top LXe detector and samples of the LUX plastics to demonstrate that the injection and removal could be done successfully. These experiments are described in the appendix.

An initial tritium dataset of $\sim$ 7,000 fiducial events was obtained by LUX in August of 2013, and the results were reported in Ref. \cite{lux-results}. Subsequently, in December 2013, we injected additional activity with a higher rate and obtained a fiducial tritium dataset of over 150,000 events. This dataset is used to calibrate the LUX ER band in Ref. \cite{lux-reanalysis}. Except where otherwise noted, in this article we report results from the larger December 2013 dataset.