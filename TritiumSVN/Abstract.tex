 \begin{abstract}

We present measurements of the electron-recoil (ER) response of the LUX dark matter detector based upon 170,000 highly pure and spatially uniform fiducial tritium decays. We reconstruct the tritium energy spectrum using the combined energy model and find good agreement with expectations. We report the average charge and light yields of ER events in liquid xenon at 180 V/cm and 105 V/cm and compare the results to the NEST model. We also measure the mean charge recombination fraction as its fluctuations, and we investigate the location and width of the LUX ER band. These results provide input to a re-analysis of the LUX Run3 WIMP search .

\end{abstract}

